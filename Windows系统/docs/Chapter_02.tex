\chapter{相关工作}
简述已有的360度视频传输算法和已有的基于MAB的360度视频算法,同时提出基于MAB的360度视频算法的缺陷。

\section{已有的360度视频传输算法}
当前已有的360度视频传输算法可以分为以下三种。


第一种是使用系统工程的方法优化360度视频传输。Xing Liu等\cite{ref2}提出的研究使用跨学科方法优化360°视频流。包括:(1)创造性地将视频编码社区开发的现有技术应用于新环境。(2)使用大数据和众包来增加流算法的智能;(3)利用跨层知识来促进多个网络上的内容分发。(4)在优化广播方和观众方的360°实时视频流方面提供了新的见解。(5)将创新集成到Sperke中,这是一个具有各种系统级优化的整体系统。总之,这是一种全方位优化360度视频传输的方法。


第二种是对历史数据进行大规模训练后试图对FoV进行预测。训练的数据有当前用户的头部运动轨迹、当前播放的视频的其他用户的头部运动轨迹和视频内容。相关的研究有:Zhang等人在\cite{ref1}中提出了一种深度强化学习(DRL)算法,根据预测的FoV和带宽学习选择最优的速率进行传输。Xie等人在\cite{ref6}中提出了一种算法,通过比较用户和不同类别的其他用户的历史FoV来识别用户的类别,从而最大化用户的视频质量。这种类型的算法在实际表现上效果不错,但是最大的问题是需要大量数据进行训练。


第三种是对传输系统建模为多臂老虎机问题(MAB),然后采用针对MAB问题的方法进行解决,并且针对360度视频传输的特点对传统解决方法进行优化,比如采用两层反馈而不是单反馈。更详细的论述见下一小节。


\section{已有的基于MAB的360度视频传输算法及缺陷}
MAB的解决方法有很多,比如A-B test、epsilon-greedy算法、UCB算法、Thompson Sampling算法等。目前已有的研究中有对UCB算法进行改进的解决方法。Harsh Gupta等人在\cite{ref7} 中提出两层反馈的MAB模型,开发了KL-UCB算法的一个重要变体,该算法有效地利用了两级反馈。论文进一步证明了它渐近匹配基本下界,这意味着它的渐近最优性。并且论文的实验结果表明,与经典的单反馈KL-UCB算法相比,该算法具有更好的性能。

在实际效果测试中,Thompson Sampling算法的改进版本表现出来的性能比UCB算法更好。UCB算法的改进版本由J. Chen等人\cite{ref8}提出,该算法有效地利用了两级反馈信息,并且证明了它的性能比单级反馈信息的性能要好得多。但是该算法未能考虑到信道容量的复杂性,测试只在单一不变的信道容量下进行。本论文进一步提出了在周期性变化的信道容量下,可以采用置零或者带有折扣系数的滑动窗口的方法来优化双反馈Thompson Sampling算法的性能。

\section{本章小结}
360度视频传输算法大致分为三种,第一种是使用系统方法进行优化。第二种方法使用大量数据进行训练,从而提高预测FoV的准确率。第三种基于MAB模型。在已有的基于MAB的360度视频传输算法研究中,表现优异的双反馈Thompson Sampling的信道条件过于单一,未能模拟出真实信道。因此本论文提出针对周期性变化的信道的双反馈Thompson Sampling的改进,包括置零或者带有折扣系数的滑动窗口的方法等方法。